\documentclass{article}
\title{tic-tac-toe Code Description}
\date{}
\author{Matteo Sepa, Daniel Schiop, Lorenzo Muzaka, Samuel Hofer \\ Siam Islam Shomapto}

% packages include
\usepackage{hyperref}
\usepackage[T1]{fontenc}

\begin{document}
\maketitle
\pagenumbering{gobble}

\newpage
\tableofcontents
\newpage
\pagenumbering{arabic}

% server
\section{Server}
\emph{Questa pagina descrive le classi ed i metodi fondamentali del modulo Server}

% classi
\subsection{Classi}
% socket connection
\subsubsection{SocketConnection}
La classe \textbf{SocketConnection} gestisce la connessione attraverso socket:\newline\newline
La classe specifica porta ed IP del server:

% code snippet
\begin{verbatim}
    # Server IP and port
    IP = "127.0.0.1"
    PORT = 12345
\end{verbatim}
Poi dichiara il tipo di socket e le opzioni di connessione:

% code snippet
\begin{verbatim}
    # Socket type and options
    server_socket = socket.socket(family=socket.AF_INET, type=socket.SOCK_STREAM)
    server_socket.setsockopt(socket.SOL_SOCKET, socket.SO_REUSEADDR, 1)

    server_socket.bind((IP, PORT))
    server_socket.listen()
\end{verbatim}

\newpage
\subsection{Metodi}

% broadcast
\subsubsection{broadcast()}
Il metodo \textbf{broadcast()} invia un messaggio di conferma a tutti i client connessi:

% code snippet
\begin{verbatim}
    def broadcast(self, message, client_socket):
        """Metodo che invia un messaggio in broadcast a gli host connessi"""
        # Send messages to all clients except to the original sender
        for client in self.clients.keys():
            if client is not client_socket:
                client.send(message.encode("utf-8"))
\end{verbatim}

% handle
\subsubsection{handle()}
Il metodo \textbf{handle()} viene chiamata per ogni client connesso e gestisce le connessioni 
con i client

% code snippet
\begin{verbatim}
    def handle(self, client_socket):
        """Metodo che gestisce la connessione"""
\end{verbatim}

% receive
\subsubsection{receive()}
Il metodo \textbf{receive()} gestisce le connessioni, i turni di gioco ed i thread per ogni client.\\

Il metodo inizializza la connessione, notifica il client e setta il nickname al client:

% code snippet
\begin{verbatim}
    def receive(self):
        """Metodo che gestisce le connessioni e fa partire il thread"""
        global turn
        while True:
            # Accept Connection
            client_socket, address = self.server_socket.accept()
            print("Connected with {}".format(str(address)))
            client_socket.send(self.turn.encode('utf-8'))
            nickname = f'player {self.turn}'
\end{verbatim}

Poi setta il turno della partita:

% code snippet
\begin{verbatim}
    if self.turn == "x":
        self.turn = "o"
    elif self.turn == "o":
        self.turn = "x"
    self.clients.update({client_socket: nickname})
\end{verbatim}

In fine avvia il thread:

\begin{verbatim}
    thread = threading.Thread(target=self.handle, args=(client_socket,))
    thread.start()
\end{verbatim}

\section{Client}
\emph{Questa pagina descrive le classi ed i metodi fondamentali del modulo Server}

\subsection{Classi}

\subsubsection{SocketChat}
La classe \textbf{SocketChat} gestisce le connessioni al socket
\newline\newline
Metodo che inizializza la connessione:

\begin{verbatim}
    def __init__(self):
        """Metodo che inizializza la connessione"""
        self.nickname = ""
        # Server Ip and Port
        self.IP = "127.0.0.1"
        self.PORT = 12345
        self.client_socket = socket.socket(
            family=socket.AF_INET, type=socket.SOCK_STREAM
        )
\end{verbatim}

\subsubsection{Game}
La classe \textbf{Game} contiene la logica di gioco
\newline\newline
L'attributo \emph{winning\char`_states} contiene le combinazioni che determinano la vincita della partita:

\begin{verbatim}
    winning_states = [
        [(0, 0), (0, 1), (0, 2)],
        [(1, 0), (1, 1), (1, 2)],
        [(2, 0), (2, 1), (2, 2)],
        [(0, 0), (1, 0), (2, 0)],
        [(0, 1), (1, 1), (2, 1)],
        [(0, 2), (1, 2), (2, 2)],
        [(0, 0), (1, 1), (2, 2)],
        [(0, 2), (1, 1), (2, 0)],
    ]
\end{verbatim}

Inizializzazione della classe:

\begin{verbatim}
    def __init__(self):
        super().__init__()
        self.player = ""
        self.turn = "x"
        self.initUI()
        self.p1_score = 0
        self.p2_score = 0
\end{verbatim}

\subsection{Metodi}

\subsubsection{receive()}
Il metodo \textbf{receive()} della classe \textbf{SocketChat} riceve il messaggio dal server:

\begin{verbatim}
    def receive(self):
        """Metodo che riceve il messaggio dal server"""
        message = self.client_socket.recv(1024).decode("utf-8")
        return message
\end{verbatim}

\subsubsection{write()}
Il metodo \textbf{write()} della classe \textbf{SocketChat} invia il messaggio di conferma al server:

\begin{verbatim}
    def write(self, msg: str):
        message = msg
        self.client_socket.send(message.encode("utf-8"))
\end{verbatim}
\newpage

\subsubsection{initUI()}
Il metodo \textbf{initUI} della classe \textbf{Game} inizializza l'interfaccia grafica:

\begin{verbatim}
    def initUI(self):
        """Metodo che inizializza l'interfaccia grafica"""
        self.game_size = 3
        self.buttons = [
            [],
            [],
            [],
        ]
        grid = QGridLayout()
        self.setLayout(grid)
\end{verbatim}

Poi crea i pulsanti:

\begin{verbatim}
    for i in range(self.game_size):
            for j in range(self.game_size):
                button = QPushButton()
                button.setFixedSize(200, 200)
                button.clicked.connect(self.takeTurn(button, i, j))
                font = button.font()
                font.setPointSize(60)
                button.setFont(font)
                grid.addWidget(button, i, j)
                self.buttons[i].append(button)
\end{verbatim}

Ed infine assegna dei metodi ai pulsanti.

\subsubsection{initDatabase()}
Il metodo \textbf{initDatabase()} della classe \textbf{Game} crea la tebella contenente i risultati delle partite
nel database nel caso non esistesse.
\newline\newline
Crea una connessione con il server di MariaDB:

\begin{verbatim}
    def initDatabase(self):
        """Metodo che inizializza la connessione al database"""
        # Connect to MariaDB Platform
        conn = mariadb.connect(
            user="tictactoeuser",
            password="password",
            host="localhost",
            database="tictactoe")
        cur = conn.cursor() 
\end{verbatim}
\newpage

Poi crea la tabella nel caso non esistesse:

\begin{verbatim}
    statement = ("CREATE TABLE IF NOT EXISTS results ("
            "gameID INT NOT NULL AUTO_INCREMENT PRIMARY KEY,"
            "p1_score INT(25),"
            "p2_score INT(25),"
            "date DATE"
            ");"
            )
            cur.execute(statement)
            conn.commit()
\end{verbatim}
Ed infine chiude la connessione:

\begin{verbatim}
    conn.close()
\end{verbatim}

\subsubsection{writeToDbServer()}
Il metodo \textbf{writeToDbServer()} della classe \textbf{Game} scrive i risultati della partita nel database
\newline\newline
Crea una connessione con il server di MariaDB:

\begin{verbatim}
    def initDatabase(self):
        """Metodo che inizializza la connessione al database"""
        # Connect to MariaDB Platform
        conn = mariadb.connect(
            user="tictactoeuser",
            password="password",
            host="localhost",
            database="tictactoe")
        cur = conn.cursor() 
\end{verbatim}

Poi inserisce i dati nel database:

\begin{verbatim}
    statement = (
                "INSERT INTO results (p1_score,p2_score,date) "
                "VALUES (%s,%s,%s)"
                )
            items_to_insert = (p1_score, p2_score, date)

            cur.execute(statement, items_to_insert)
            conn.commit() 
\end{verbatim}

Ed infine chiude la connessione:

\begin{verbatim}
    conn.close()
\end{verbatim}
\newpage

\subsubsection{newGame()}
Il metodo \textbf{newGame()} della classe \textbf{Game} fa iniziare una nuova partita azzerando i componenti dell'interfaccia grafica:

\begin{verbatim}
    def newGame(self):
        """Metodo che fa iniziare una nuova partita"""
        for row in self.buttons:
            for btn in row:
                btn.setText("")
\end{verbatim}

\subsubsection{checkGame()}
Il metodo \textbf{chackGame()} della classe \textbf{Game} controlla lo stato della partita facendo riferimento al attributo winning\char`_states:

\begin{verbatim}
    def checkGame(self):
        """Metodo che controlla lo stato della partita"""
        win = ""
        for win_state in Game.winning_states:
\end{verbatim}

Nel caso di vincita aumenta il punteggio del giocatore che ha vinto e richiama il metodo \textbf{newGame()}:

\begin{verbatim}
    win = state
    print(f"'{win}' vince")
    self.player_won_label.setText("{} ha vinto".format(win))
    if win == 'x':
        self.p1_score = self.p1_score + 1
    else:
        self.p2_score = self.p2_score + 1
    self.newGame()
\end{verbatim}

Stessa cosa in caso di pareggio senza però alterare il punteggio dei giocatori.

\subsubsection{otherPlayerTurn()}
Il metodo \textbf{otherPlayerturn()} della classe \textbf{Game} avvia il thread che gestisce il turno per il prossimo giocatore:

\begin{verbatim}
    def otherPlayerTurn(self):
        threading.Thread(target=self._otherPlayerTurn).start()
\end{verbatim}
\newpage

\subsubsection{toggleTurn()}
Il metodo \textbf{toggleTurn()} della classe \textbf{Game} alterna i turni durante la partita:

\begin{verbatim}
    def toggleTurn(self):
        """Metodo che alterna i turni durante la partita"""
        if self.turn == "x":
            self.turn = "o"
        else:
            self.turn = "x"
        self.turn_label.setText("{}\nTurn".format(self.turn))
\end{verbatim}

\subsubsection{endTurn()}
Il metodo \textbf{endTurn()} della classe \textbf{Game} fa terminare il turno in corso:

\begin{verbatim}
    def endTurn(self):
        self.toggleTurn()
        self.checkGame()
        self.otherPlayerTurn()
\end{verbatim}

\end{document}