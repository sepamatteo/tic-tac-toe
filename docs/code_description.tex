\documentclass{article}
\title{tic-tac-toe Code Description}
\date{}
\author{Matteo Sepa, Daniel Schiop, Lorenzo Muzaka, Samuel Hofer \\ Siam Islam Shomapto}

% packages include
\usepackage{hyperref}

\begin{document}
\maketitle
\pagenumbering{gobble}

\newpage
\tableofcontents
\newpage
\pagenumbering{arabic}

% server
\section{Server}
\emph{Questa pagina descrive le classi ed i metodi fondamentali del modulo Server}

% classi
\subsection{Classi}
% socket connection
\subsubsection{SocketConnection}
La classe \textbf{SocketConnection} gestisce la connessione attraverso socket:\newline\newline
La classe specifica porta ed IP del server:

% code snippet
\begin{verbatim}
    # Server IP and port
    IP = "127.0.0.1"
    PORT = 12345
\end{verbatim}
Poi dichiara il tipo di socket e le opzioni di connessione:

% code snippet
\begin{verbatim}
    # Socket type and options
    server_socket = socket.socket(family=socket.AF_INET, type=socket.SOCK_STREAM)
    server_socket.setsockopt(socket.SOL_SOCKET, socket.SO_REUSEADDR, 1)

    server_socket.bind((IP, PORT))
    server_socket.listen()
\end{verbatim}

\newpage
\subsection{Metodi}

\subsubsection{broadcast}
Il metodo \textbf{broadcast} invia un messaggio di conferma a tutti i client connessi:

\begin{verbatim}
    def broadcast(self, message, client_socket):
        """Metodo che invia un messaggio in broadcast a gli host connessi"""
        # Send messages to all clients except to the original sender
        for client in self.clients.keys():
            if client is not client_socket:
                client.send(message.encode("utf-8"))
\end{verbatim}

\subsubsection{handle}
Il metodo \textbf{handle} viene chiamata per ogni client connesso e gestisce le connessioni 
con i client

\begin{verbatim}
    def handle(self, client_socket):
        """Metodo che gestisce la connessione"""
\end{verbatim}

\subsubsection{receive}
Il metodo \textbf{receive} gestisce le connessioni, i turni di gioco ed i thread per ogni client.\\

Il metodo inizializza la connessione, notifica il client e setta il nickname al client:

\begin{verbatim}
    def receive(self):
        """Metodo che gestisce le connessioni e fa partire il thread"""
        global turn
        while True:
            # Accept Connection
            client_socket, address = self.server_socket.accept()
            print("Connected with {}".format(str(address)))
            client_socket.send(self.turn.encode('utf-8'))
            nickname = f'player {self.turn}'
\end{verbatim}

Poi setta il turno della partita:

\begin{verbatim}
    if self.turn == "x":
        self.turn = "o"
    elif self.turn == "o":
        self.turn = "x"
    self.clients.update({client_socket: nickname})
\end{verbatim}

In fine avvia il thread:

\begin{verbatim}
    thread = threading.Thread(target=self.handle, args=(client_socket,))
    thread.start()
\end{verbatim}

\end{document}