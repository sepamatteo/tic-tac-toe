\documentclass{article}
\title{tic-tac-toe User Guide}
\date{}
\author{Matteo Sepa, Daniel Schiop, Lorenzo Muzaka, Samuel Hofer \\ Siam Islam Shomapto}

% packages include
\usepackage{hyperref}
\usepackage{graphicx}
% images directory path
\graphicspath{{img}}

\begin{document}
\maketitle
\pagenumbering{gobble}

% including image
\includegraphics[scale=5.5]{tictactoe.png}
\centering

\newpage
\tableofcontents
\newpage
\pagenumbering{arabic}

% requisiti
\section{Prerequisiti}

\subsection{Lista dei requisiti}
\begin{itemize}
    \item Python 3
    \item MariaDB
\end{itemize}

\subsection{Lista librerie python}
Per installare una libreria digitare il comando: \emph{pip3 install nome-libreria}
\begin{itemize}
    \item PySide2
    \item Mariadb
\end{itemize}

% eseguire l'applicazione
\newpage
\section{Eseguire l'applicazione}
\emph{Il seguente metodo funziona esclusivamente su macchine GNU/Linux}

\subsection{Rendere lo script eseguibile}
Per eseguire il programma basta semplicemente rendere eseguibile lo script \emph{launch.sh}
eseguendo il comando: \textbf{\emph{chmod +x launch.sh}}

\subsection{Eseguire lo script}
In seguito per eseguire il programma basterà mandare in esecuzione il programma 
basterà eseguire lo script con il comando: \textbf{\emph{./launch.sh}}

% terminare la partita
\newpage
\section{Terminare la partita}
\emph{Terminare la partita permette di salvare i dati della partita sul database}

\subsection{Come terminare la partita}
Per terminare la partita basterà semplicemente clickare sul pulsante \textbf{RESET}
presente sull'interfaccia grafica.

\end{document}